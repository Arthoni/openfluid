\subsection{Simulator name}
The name (ID) of the simulator is \texttt{\FileID}.


\subsection{Simulator parameters}
The simulator ``\englishname'' must be used with the following parameters :
\vspace{1em}

\hspace{-0.5cm}
\begin{tabular}{|llcc|}
 \hline
\it Symbol & \it Name & \it Value range & \it Unit \\
 \hline
$Max Step$ & \texttt{\ParamA} & $>0$ & $-$ \\
$C$ & \texttt{\ParamB} & $>0$ & $m/s$ \\
$D$ & \texttt{\ParamC} & $>0$ & $m\up2/s$ \\
$Calib Step$ & \texttt{\ParamD} & $>0$ & $m$ \\
$RS buffer$ & \texttt{\ParamE} & $\geq 0$ & $m$ \\
\hline
\end{tabular} 
\vspace{1em}

Thus, the correct syntax to use in the \texttt{model.xml} file is illustrated hereafter.

\begin{small}
\begin{verbatim}
<simulator ID="water.surf.
           transfer-rs.hayami">
    <param name="calibstep" value="0.005" />
    <param name="maxsteps" value="100" />
    <param name="meancel" value="0.5" />
    <param name="meansigma" value="500" />
    <param name="rsbuffer" value="1" />
</simulator>
\end{verbatim}
\end{small}



\subsection{Unit properties required}
The simulator requires some geometric properties and soil characteristics. These are described in the following table.
\vspace{1em}

\hspace{-0.5cm}
\begin{tabular}{|llcc|}
 \hline
\it Symbol &\it Name & \it Value range & \it Unit \\
 \hline
$n$ & \texttt{\PropDisA} & $>0$ & $s/m\up{-1/3}$ \\
$L$ & \texttt{\PropDisB} & $>0$ & $m$ \\
$l$ & \texttt{\PropDisC} & $>0$ & $m$ \\
$H$ & \texttt{\PropDisD} & $>0$ & $m$ \\
$\beta$ & \texttt{\PropDisE} & $>0$ & $m/m$ \\
\hline
\end{tabular}
\vspace{1em}



\subsection{Variables}
Variables produced, required and updated by the simulator are listed hereafter.
\vspace{1em}

\hspace{-0.5cm}
\begin{tabular}{|lll|}
 \hline
\it Symbol & \it Name & \it Unit \\
 \hline
$Q_{SU}$ & \texttt{\VarUsedA} & $m\up{3}/s$ \\
$Q_b$ & \texttt{\VarUsedB} & $m\up{3}/s$ \\
$Q_i$ & \texttt{\VarUsedC} & $m\up{3}/s$ \\
$Q_{RS}$ & \texttt{\VarProdA} & $m\up{3}/s$ \\
$h$ & \texttt{\VarProdB} & $m$ \\
\hline
\end{tabular} 
\vspace{1em}
